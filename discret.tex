\documentclass{article}

\usepackage[T1]{fontenc}
\usepackage[french]{babel}
\usepackage{amsmath, amssymb, amsthm, mathtools}

\newcommand{\R}{\mathbb{R}}
\newcommand{\Z}{\mathbb{Z}}
\newtheorem*{thm}{Théorème}

\setlength{\parindent}{0pt}
\setlength{\parskip}{1em}

\begin{document}
\begin{thm}
    Soit \( E \subset \R^n \) un sous-espace vectoriel de dimension \( m \leq n \). Soit \( G \leq E \) un sous-groupe discret de \( E \). Alors, \( G \) est isomorphe à \( \Z^r \) pour un certain \( r \leq m \).
\end{thm}

\begin{proof}
    On démontre par récurrence en $m = \dim E$. Si $m = 0$, alors $E = \{0\}$ et $G = \{0\}$, donc $G \cong \Z^0$. Sinon, supposons que le théorème est vrai pour tout sous-espace vectoriel de dimension inférieure à $m$. Si $G = \{0\}$, alors $G \cong \Z^0$. Supposons donc que $G \neq \{0\}$. Définissons
    \[
    a \coloneq \inf \{ |g| \mid g \in G \setminus \{0\} \}.
    \]
    Comme $G$ est discret, il existe un $\varepsilon > 0$ tel que $B(0, \varepsilon) \cap G = \{0\}$. Par conséquent, $a \geq \varepsilon > 0$. Pour $g, h \in G$ distincts, les boules $B(g, a/2)$ et $B(h, a/2)$ sont disjointes. Cela signifie que $(G \setminus \{0\}) \cap B(0, 2a)$ est fini et non vide. On peut donc trouver un élément $\omega \in G$ tel que $|\omega| = a$.
    
    On affirme que $G \cap \R \omega = \Z \omega$. En effet, si $g \in G \cap \R \omega$, alors $g = \lambda \omega$ pour un certain $\lambda \in \R$. Puis, $g' = g - \lfloor \lambda \rfloor \omega \in G$ et $|g'| < a$, donc $g' = 0$, ce qui implique $g = \lfloor \lambda \rfloor \omega \in \Z \omega$.

    Soit $F$ le complément orthogonal de $\omega$ dans $E$, et $\pi \colon E \to F$ la projection orthogonale sur $F$. On affirme que $\pi(G)$ est discret. Supposons le contraire. Alors, il existe une suite infinie d'éléments distincts $h_n \in \pi(G)$ avec $|h_n| < a$. Chaque $h_n$ provient d'un élément $g_n \in G$ tel que $\pi(g_n) = h_n$. Chaque $g_n$ peut s'écrire $g_n = \lambda_n \omega + h_n$. Considérant $g_n' = g_n - \lfloor \lambda_n \rfloor \omega \in G$ avec $|g_n'| < 2a$, nous obtiendrons une suite infinie d'éléments distincts dans $B(0, 2a)$, ce qui est impossible. Donc, $\pi(G)$ est discret.

    Par l'hypothèse de récurrence, $\pi(G)$ est isomorphe à $\Z^r$ pour un certain $r \leq m - 1$. Soit $\varphi \colon \pi(G) \to \Z^r$ un isomorphisme, et soient $\tilde \omega_1, \ldots, \tilde \omega_r \in \pi(G)$ tels que $\varphi(\tilde \omega_i) = e_i$, où $(e_1, \ldots, e_r)$ est la base canonique de $\Z^r$. Soient $\omega_1, \ldots, \omega_r \in G$ tels que $\pi(\omega_i) = \tilde \omega_i$. Considérons $\psi \colon \pi(G) \to G$ l'homomorphisme défini par $\psi(\tilde \omega_i) = \omega_i$. On observe que $\pi(\psi(\tilde \omega_i)) = \tilde \omega_i$ pour tout $i$, ce qui montre que $\pi \circ \psi = \operatorname{id}_{\pi(G)}$.

    De plus, pour tout $g \in G$, on a $\pi(\psi(\pi(g))) = \pi(g)$, donc $\psi(\pi(g)) - g \in \ker \pi = \R \omega$. Comme $\psi(\pi(g)) - g \in G$, on a $\psi(\pi(g)) - g \in G \cap \R \omega = \Z \omega$, donc $g = \psi(\pi(g)) + x \omega$ pour un certain $x \in \Z$. Ce $x$ est unique, car si $g = \psi(\pi(g)) + x \omega = \psi(\pi(g)) + y \omega$, alors $x\omega = y\omega$, donc $x = y$.

    Définissons alors l'application $\Phi \colon G \to \Z^r \times \Z$ par $\Phi(g) = (\varphi(\pi(g)), x)$, où $g = \psi(\pi(g)) + x \omega$. Il est facile de vérifier que $\Phi$ est un isomorphisme.
\end{proof}

\end{document}